\documentclass[]{article}
\usepackage{lmodern}
\usepackage{amssymb,amsmath}
\usepackage{ifxetex,ifluatex}
\usepackage{fixltx2e} % provides \textsubscript
\ifnum 0\ifxetex 1\fi\ifluatex 1\fi=0 % if pdftex
  \usepackage[T1]{fontenc}
  \usepackage[utf8]{inputenc}
\else % if luatex or xelatex
  \ifxetex
    \usepackage{mathspec}
  \else
    \usepackage{fontspec}
  \fi
  \defaultfontfeatures{Ligatures=TeX,Scale=MatchLowercase}
\fi
% use upquote if available, for straight quotes in verbatim environments
\IfFileExists{upquote.sty}{\usepackage{upquote}}{}
% use microtype if available
\IfFileExists{microtype.sty}{%
\usepackage{microtype}
\UseMicrotypeSet[protrusion]{basicmath} % disable protrusion for tt fonts
}{}
\usepackage[margin=1in]{geometry}
\usepackage{hyperref}
\hypersetup{unicode=true,
            pdftitle={36-315 Lab 01},
            pdfauthor={Eu Jing Chua},
            pdfborder={0 0 0},
            breaklinks=true}
\urlstyle{same}  % don't use monospace font for urls
\usepackage{color}
\usepackage{fancyvrb}
\newcommand{\VerbBar}{|}
\newcommand{\VERB}{\Verb[commandchars=\\\{\}]}
\DefineVerbatimEnvironment{Highlighting}{Verbatim}{commandchars=\\\{\}}
% Add ',fontsize=\small' for more characters per line
\usepackage{framed}
\definecolor{shadecolor}{RGB}{248,248,248}
\newenvironment{Shaded}{\begin{snugshade}}{\end{snugshade}}
\newcommand{\AlertTok}[1]{\textcolor[rgb]{0.94,0.16,0.16}{#1}}
\newcommand{\AnnotationTok}[1]{\textcolor[rgb]{0.56,0.35,0.01}{\textbf{\textit{#1}}}}
\newcommand{\AttributeTok}[1]{\textcolor[rgb]{0.77,0.63,0.00}{#1}}
\newcommand{\BaseNTok}[1]{\textcolor[rgb]{0.00,0.00,0.81}{#1}}
\newcommand{\BuiltInTok}[1]{#1}
\newcommand{\CharTok}[1]{\textcolor[rgb]{0.31,0.60,0.02}{#1}}
\newcommand{\CommentTok}[1]{\textcolor[rgb]{0.56,0.35,0.01}{\textit{#1}}}
\newcommand{\CommentVarTok}[1]{\textcolor[rgb]{0.56,0.35,0.01}{\textbf{\textit{#1}}}}
\newcommand{\ConstantTok}[1]{\textcolor[rgb]{0.00,0.00,0.00}{#1}}
\newcommand{\ControlFlowTok}[1]{\textcolor[rgb]{0.13,0.29,0.53}{\textbf{#1}}}
\newcommand{\DataTypeTok}[1]{\textcolor[rgb]{0.13,0.29,0.53}{#1}}
\newcommand{\DecValTok}[1]{\textcolor[rgb]{0.00,0.00,0.81}{#1}}
\newcommand{\DocumentationTok}[1]{\textcolor[rgb]{0.56,0.35,0.01}{\textbf{\textit{#1}}}}
\newcommand{\ErrorTok}[1]{\textcolor[rgb]{0.64,0.00,0.00}{\textbf{#1}}}
\newcommand{\ExtensionTok}[1]{#1}
\newcommand{\FloatTok}[1]{\textcolor[rgb]{0.00,0.00,0.81}{#1}}
\newcommand{\FunctionTok}[1]{\textcolor[rgb]{0.00,0.00,0.00}{#1}}
\newcommand{\ImportTok}[1]{#1}
\newcommand{\InformationTok}[1]{\textcolor[rgb]{0.56,0.35,0.01}{\textbf{\textit{#1}}}}
\newcommand{\KeywordTok}[1]{\textcolor[rgb]{0.13,0.29,0.53}{\textbf{#1}}}
\newcommand{\NormalTok}[1]{#1}
\newcommand{\OperatorTok}[1]{\textcolor[rgb]{0.81,0.36,0.00}{\textbf{#1}}}
\newcommand{\OtherTok}[1]{\textcolor[rgb]{0.56,0.35,0.01}{#1}}
\newcommand{\PreprocessorTok}[1]{\textcolor[rgb]{0.56,0.35,0.01}{\textit{#1}}}
\newcommand{\RegionMarkerTok}[1]{#1}
\newcommand{\SpecialCharTok}[1]{\textcolor[rgb]{0.00,0.00,0.00}{#1}}
\newcommand{\SpecialStringTok}[1]{\textcolor[rgb]{0.31,0.60,0.02}{#1}}
\newcommand{\StringTok}[1]{\textcolor[rgb]{0.31,0.60,0.02}{#1}}
\newcommand{\VariableTok}[1]{\textcolor[rgb]{0.00,0.00,0.00}{#1}}
\newcommand{\VerbatimStringTok}[1]{\textcolor[rgb]{0.31,0.60,0.02}{#1}}
\newcommand{\WarningTok}[1]{\textcolor[rgb]{0.56,0.35,0.01}{\textbf{\textit{#1}}}}
\usepackage{graphicx,grffile}
\makeatletter
\def\maxwidth{\ifdim\Gin@nat@width>\linewidth\linewidth\else\Gin@nat@width\fi}
\def\maxheight{\ifdim\Gin@nat@height>\textheight\textheight\else\Gin@nat@height\fi}
\makeatother
% Scale images if necessary, so that they will not overflow the page
% margins by default, and it is still possible to overwrite the defaults
% using explicit options in \includegraphics[width, height, ...]{}
\setkeys{Gin}{width=\maxwidth,height=\maxheight,keepaspectratio}
\IfFileExists{parskip.sty}{%
\usepackage{parskip}
}{% else
\setlength{\parindent}{0pt}
\setlength{\parskip}{6pt plus 2pt minus 1pt}
}
\setlength{\emergencystretch}{3em}  % prevent overfull lines
\providecommand{\tightlist}{%
  \setlength{\itemsep}{0pt}\setlength{\parskip}{0pt}}
\setcounter{secnumdepth}{0}
% Redefines (sub)paragraphs to behave more like sections
\ifx\paragraph\undefined\else
\let\oldparagraph\paragraph
\renewcommand{\paragraph}[1]{\oldparagraph{#1}\mbox{}}
\fi
\ifx\subparagraph\undefined\else
\let\oldsubparagraph\subparagraph
\renewcommand{\subparagraph}[1]{\oldsubparagraph{#1}\mbox{}}
\fi

%%% Use protect on footnotes to avoid problems with footnotes in titles
\let\rmarkdownfootnote\footnote%
\def\footnote{\protect\rmarkdownfootnote}

%%% Change title format to be more compact
\usepackage{titling}

% Create subtitle command for use in maketitle
\providecommand{\subtitle}[1]{
  \posttitle{
    \begin{center}\large#1\end{center}
    }
}

\setlength{\droptitle}{-2em}

  \title{36-315 Lab 01}
    \pretitle{\vspace{\droptitle}\centering\huge}
  \posttitle{\par}
    \author{Eu Jing Chua}
    \preauthor{\centering\large\emph}
  \postauthor{\par}
      \predate{\centering\large\emph}
  \postdate{\par}
    \date{Due Saturday, August 31, 2019 (6:30pm)}


\begin{document}
\maketitle

\hypertarget{lab-01-hello-world-introduction-to-r-rstudio-rmarkdown}{%
\subsection{Lab 01: Hello, world! Introduction to R, RStudio, \&
RMarkdown}\label{lab-01-hello-world-introduction-to-r-rstudio-rmarkdown}}

\textbf{\emph{General instructions for all assignments}}:

\begin{itemize}
\item
  Use this file as the template for your submission. Delete the
  unnecessary text (e.g.~this text, the problem statements, etc). That
  said, keep the nicely formatted ``Problem 1'', ``Problem 2'', ``a.'',
  ``b.'', etc
\item
  Upload a single \texttt{R} Markdown file (named as:
  {[}AndrewID{]}-315-Lab01.Rmd -- e.g.~``larry-315-Lab01.Rmd'') to the
  Lab 01 submission section on Canvas. You do not need to upload the
  .html file.
\item
  The instructor and TAs will run your .Rmd file on their computer.
  \textbf{If your .Rmd file does not knit on our computers, you will be
  automatically get 0 points.}
\item
  Your file should contain the code to answer each question in its own
  code block. Your code should produce plots/output that will be
  automatically embedded in the output (.html) file
\item
  Each answer must be supported by written statements (unless otherwise
  specified)
\item
  Include the name of anyone you collaborated with at the top of the
  assignment
\item
  Include the style guide you used at the top of the assignment
\end{itemize}

\begin{center}\rule{0.5\linewidth}{\linethickness}\end{center}

\begin{center}\rule{0.5\linewidth}{\linethickness}\end{center}

Todays lab will introduce different aspects of \texttt{R}. We will
reference a long handout on the Canvas site under Labs
(Rintro-wasserman.pdf).

\begin{center}\rule{0.5\linewidth}{\linethickness}\end{center}

\begin{center}\rule{0.5\linewidth}{\linethickness}\end{center}

\hypertarget{this-week-only-the-assignment-is-due-on-saturday-at-630pm.-normally-lab-assignments-will-be-due-on-fridays-at-630pm.}{%
\subsection{This week only, the assignment is due on Saturday at 6:30pm.
Normally, lab assignments will be due on Fridays at
6:30pm.}\label{this-week-only-the-assignment-is-due-on-saturday-at-630pm.-normally-lab-assignments-will-be-due-on-fridays-at-630pm.}}

\begin{center}\rule{0.5\linewidth}{\linethickness}\end{center}

\begin{center}\rule{0.5\linewidth}{\linethickness}\end{center}

Here is a graph"

\begin{figure}
\centering
\includegraphics{https://pbs.twimg.com/media/CqAAH9yW8AA2Jj6.jpg}
\caption{Figure caption}
\end{figure}

\begin{center}\rule{0.5\linewidth}{\linethickness}\end{center}

\begin{center}\rule{0.5\linewidth}{\linethickness}\end{center}

\hypertarget{before-you-start}{%
\section{Before You Start}\label{before-you-start}}

This lab is an introduction to \texttt{R}. This will be review for many
of you, and it will be brand new to many others. \textbf{If you have
trouble completing the lab on time, don't worry.}

Let your TAs know what you're having trouble with, and they will help
you.

This lab is for your benefit; we're most concerned about you learning
and understanding \texttt{R} and \texttt{R\ Markdown}. Do not worry
about your grade on this lab -- we will be lenient.

\begin{center}\rule{0.5\linewidth}{\linethickness}\end{center}

\begin{center}\rule{0.5\linewidth}{\linethickness}\end{center}

\hypertarget{problem-1}{%
\section{Problem 1}\label{problem-1}}

(15 points)

\textbf{Storing Your Work and Storing Your \texttt{R} Commands}:

First, open the RStudio program on your computer.

With R Markdown, you can simultaneously store your commands, execute
your commands, and generate an output file automatically. Because of
this, we will use R Markdown files (.Rmd) to store our code and answers
for all Lab and Homework assignments.

Use the Lab01.Rmd file as a template for your submission. Click ``Knit
HTML'' at the top (or Command+Shift+K on Mac). This should produce an
HTML page that executes the code you write. You can edit this .Rmd file
to produce your lab submission.

Notice how the text you write in the .Rmd file shows up in the output
HTML file each time you click ``Knit HTML''.

For more tips/tricks on how to format things in R Markdown,
\href{https://www.rstudio.com/wp-content/uploads/2015/02/rmarkdown-cheatsheet.pdf}{see
here}

\textbf{Code Blocks}:

When you open a new R Markdown file, you should see a block of code
(beginning with three
\href{https://en.wikipedia.org/wiki/Grave_accent}{grave accent marks} --
similar to apostrophes, but located next to the 1 key on a standard
keyboard -- then an ``\texttt{\{r\}}'', then three more accent marks).
This block should have a light gray background color in R Studio. This
is a code block! You can type commands into this block, and they will be
executed by \texttt{R} and included in your output.

Let's test out the following command, the following command:

\begin{Shaded}
\begin{Highlighting}[]
\CommentTok{# This is a comment.}
\KeywordTok{print}\NormalTok{(}\StringTok{"Hello, World!"}\NormalTok{)}
\end{Highlighting}
\end{Shaded}

\begin{verbatim}
## [1] "Hello, World!"
\end{verbatim}

Change the above code so that it outputs the following text: ``Hello,
World! My name is {[}your name{]}.''

\textbf{Code Comments}:

Comments should be used frequently when writing code to give insight
into what each piece of code is doing. To add a comment to your code,
start a new line with the \texttt{\#} symbol.

Change the existing comment in your first code block so that it says ``I
am printing `Hello, World!' in R Markdown''.

Add another comment that says what your major is.

Notice that the text in the comments shows up in the code block in the
output file, but not in the actual output when you click ``Knit HTML''.
Comments exist to help you and others who read your code.

\textbf{Where is my output file?}

Find where you stored your {[}AndrewID{]}-315-Lab01.Rmd file on your
computer.

In the same directory, there should be a file called
{[}AndrewID{]}-315-Lab01.html.

Open it. It should automatically open in a browser, and it should
contain the output.

\begin{center}\rule{0.5\linewidth}{\linethickness}\end{center}

\begin{center}\rule{0.5\linewidth}{\linethickness}\end{center}

\hypertarget{problem-2}{%
\section{Problem 2}\label{problem-2}}

(15 points)

\textbf{Getting Started}: Create a single code block that contains R
commands to do all of the following:

\begin{enumerate}
\def\labelenumi{\alph{enumi}.}
\tightlist
\item
  \((6+3)*4 - 5\)
\item
  \(4^2\)
\item
  \(e^{-5}\) (hint: type \texttt{help(exp)} at the command line in R
  Studio
\item
  Let \(y = 8\). Let \(x = 5-6y\). Print \(x\).
\item
  Repeat (d) for \(y = 0\), \(y = 1\), and \(y = 5/6\).
\end{enumerate}

\begin{center}\rule{0.5\linewidth}{\linethickness}\end{center}

\begin{center}\rule{0.5\linewidth}{\linethickness}\end{center}

\hypertarget{problem-3}{%
\section{Problem 3}\label{problem-3}}

(15 points)

\textbf{Built-in Help and Documentation}: Using the \texttt{help()} and
\texttt{help.search()} functions at the command line, use R to do the
following:

\begin{enumerate}
\def\labelenumi{\alph{enumi}.}
\tightlist
\item
  Find the help documentation for the \texttt{quantile} function. This
  function takes a vector of numbers and computes quantiles for the
  vectors. What is the description of the \texttt{probs} argument?
\item
  Find the help documentation for the \texttt{mean} function. This
  function takes a vector of numbers and computes their average. What is
  the example code at the bottom of the help page?
\item
  Use the help pages to find the name of the function in \texttt{R} that
  finds the standard deviation of a vector, and apply it to the data
  from b.
\end{enumerate}

Note that because everything is online, you can use online search
engines to achieve many of these same goals. Feel free to do so for all
of your future assignments.

\begin{center}\rule{0.5\linewidth}{\linethickness}\end{center}

\begin{center}\rule{0.5\linewidth}{\linethickness}\end{center}

\hypertarget{problem-4}{%
\section{Problem 4}\label{problem-4}}

(15 points)

\textbf{Loading a library in R}: In \texttt{R}, there are many libraries
or packages/groups of programs that are not permanently stored in
\texttt{R}, so we have to load them when we want to use them. We can
load the library into \texttt{R} by typing
\texttt{library(library-name)} at the command line. (Sometimes we need
to download the library first; more on this later.)

\begin{enumerate}
\def\labelenumi{\alph{enumi}.}
\tightlist
\item
  Load the \texttt{MASS} library into R. Open the help documentation for
  the MASS package. What is the official name of this \texttt{MASS}
  package? It may help to use \texttt{library(help\ =\ MASS)} to solve
  this problem.
\item
  Load the \texttt{datasets} library. Find the help documentation for
  its \texttt{trees} dataset. Describe this dataset using information
  from the help pages.
\item
  Load the \texttt{graphics} library into \texttt{R} and open its help
  documentation. This library is full of graphics/visualization
  functions that we may use in this class. Find a function that creates
  a 1-D Scatterplot. Describe its argument \texttt{x}.
\end{enumerate}

\begin{center}\rule{0.5\linewidth}{\linethickness}\end{center}

\begin{center}\rule{0.5\linewidth}{\linethickness}\end{center}

\hypertarget{problem-5}{%
\section{Problem 5}\label{problem-5}}

(15 points)

\textbf{Installing Packages}

In this class, we'll use many \texttt{R} libraries (``packages'') that
are not currently installed on your computer or the lab computers. There
are some security measures in place to prevent you from installing
packages to CMU-wide repository of packages. Luckily, there are some
ways around this.

\begin{enumerate}
\def\labelenumi{\alph{enumi}.}
\tightlist
\item
  At the command line / Console (NOT in a code block), type
  \texttt{install.packages("tidyverse")}. What happens?
\end{enumerate}

If you're using one of the CMU cluster computers, the package may not
install. This happens because CMU does not allow us to install new
packages to the default location. As a result, we have to specify a new
directory where we can install new \texttt{R} packages.

If the package installed with no issues, you can skip the following
parts! If you could not install the package, take the following steps:

\begin{enumerate}
\def\labelenumi{\alph{enumi}.}
\setcounter{enumi}{1}
\tightlist
\item
  Create a new directory on your computer called ``36-315'', and create
  a new sub-directory called ``Packages''. The filepath to this
  directory should be something like:
\end{enumerate}

\begin{itemize}
\tightlist
\item
  ``C:/Users/YourName/Desktop/36-315/Packages'' if you use Windows
\item
  ``/Users/YourName/Desktop/36-315/Packages'' if you use Mac
\item
  ``\ldots{}'' if you are using the CMU cluster computers
\end{itemize}

\begin{enumerate}
\def\labelenumi{\alph{enumi}.}
\setcounter{enumi}{2}
\item
  In a code block, store the filepath in an object called
  \texttt{package\_path},
  e.g.~\texttt{package\_path\ \textless{}-\ "/Users/YourName/Desktop/36-315/Packages"}.
  Repeat this at the command line / Console as well.
\item
  In the same code block, include the following line of code:
  \texttt{.libPaths(c(package\_path,\ .libPaths()))}
\item
  At the command line / Console (NOT in a code block), type
  \texttt{install.packages("tidyverse",\ lib\ =\ package\_path)}. This
  should install the \texttt{tidyverse} package.
\item
  Finally, add \texttt{library(tidyverse)} to your code block from
  above. This should load the \texttt{tidyverse} package so that you can
  use it for the next problem.
\end{enumerate}

Important note: Never install new packages in a code block. Always
install new packages at the command line / Console. That is, the
\texttt{install.packages()} function should never be in your submitted
code. The \texttt{library()} function, however, should be in most of
your submitted code.

\begin{center}\rule{0.5\linewidth}{\linethickness}\end{center}

\begin{center}\rule{0.5\linewidth}{\linethickness}\end{center}

\hypertarget{problem-6}{%
\section{Problem 6}\label{problem-6}}

(25 points)

\textbf{Reading, Manipulating, and Plotting Data}

We will use the \texttt{read\_csv()} function to read in a dataset from
the internet. We'll use the dataset at
{[}\url{https://raw.githubusercontent.com/mateyneykov/315_code_data/master/data/bridges-pgh.csv}{]}
for this lab.

(Note that this data is in a GitHub repository called
``315\_code\_and\_data''. We'll use this repository for sharing various
datasets and code snippets throughout the semester.)

All code is provided for you for the following problems -- just
uncomment the code in the code block below:

\begin{enumerate}
\def\labelenumi{\alph{enumi}.}
\tightlist
\item
  Load the \texttt{tidyverse} library.
\item
  Read in the ``Pittsburgh Bridges'' dataset.
\item
  Create a new column in the dataset that indicates whether the bridge
  crosses the Allegheny River (``yes'' or ``no'').
\item
  Create a new column in the dataset that indicates whether the bridge
  is at least 3000 feet long (``long'' or ``short'').
\item
  Create a well-labled bar plot of the \texttt{river} variable.
\end{enumerate}

Although we provided the code for you here, it is important that you
take some time to understand what the code does! Study it in detail, and
be prepared to do similar tasks in subsequent assignments.

\begin{Shaded}
\begin{Highlighting}[]
\CommentTok{# Load the tidyverse library}
\CommentTok{#library(tidyverse)}

\CommentTok{# Read in the data}
\CommentTok{#bridges <- read_csv("https://raw.githubusercontent.com/mateyneykov/315_code_data/master/data/bridges-pgh.csv")}



\CommentTok{# Create the new variables}
\CommentTok{#bridges <- mutate(bridges, }
\CommentTok{#   over_allegheny = ifelse(river ==  "A", "yes", "no"),}
\CommentTok{#   length_binary = ifelse(length >=  3000, "long", "short"))}

\CommentTok{# Create the bar plot}
\CommentTok{#ggplot(bridges, aes(x = river)) + }
\CommentTok{# geom_bar(fill = "darkblue") + }
\CommentTok{# labs(}
\CommentTok{# title = "Number of Bridges in Pittsburgh by River",}
\CommentTok{# subtitle = "A = Allegheny River, M = Monongahela River, O = Ohio River",}
\CommentTok{# x = "River",}
\CommentTok{# y = "Number of Bridges",}
\CommentTok{# caption = "Source: Pittsburgh Bridges Dataset"}
\CommentTok{# )}
\end{Highlighting}
\end{Shaded}

\begin{center}\rule{0.5\linewidth}{\linethickness}\end{center}

\begin{center}\rule{0.5\linewidth}{\linethickness}\end{center}

\begin{center}\rule{0.5\linewidth}{\linethickness}\end{center}

\begin{center}\rule{0.5\linewidth}{\linethickness}\end{center}

\begin{center}\rule{0.5\linewidth}{\linethickness}\end{center}

\begin{center}\rule{0.5\linewidth}{\linethickness}\end{center}


\end{document}
